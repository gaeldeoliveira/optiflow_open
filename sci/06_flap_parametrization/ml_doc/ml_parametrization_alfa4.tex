%% LyX 2.2.3 created this file.  For more info, see http://www.lyx.org/.
%% Do not edit unless you really know what you are doing.
\documentclass[english]{article}
\usepackage[T1]{fontenc}
\usepackage[latin9]{inputenc}
\usepackage{amsmath}
\usepackage{amssymb}
\usepackage{babel}
\begin{document}

\section{Quantification of code accuracy}

The accuracy of numerical predictions is quantified with a Least Squares
norm regularized for the extent of data coverage. Here is an example
for lift polars, where $C_{l}^{exp}$ denotes the measured lift coefficient
and $C_{l}^{sim}$ denotes the simulated value:
\[
\epsilon_{i}^{C_{l}}=\left(\alpha_{max}-\alpha_{min}\right)^{-1}\left(\int_{\alpha_{min}}^{\alpha_{max}}\left(C_{l}^{sim}-C_{l}^{exp}\right)^{2}d\alpha\right)^{\frac{1}{2}}
\]
The same approach is adopted for drag and moment coefficient polars,
when available. A similar norm is adopted for boundary layer runs,
but the integration proceeds along the streamwise coordinate until
the end of the dataset at $x=L$:
\[
\epsilon_{i}^{\theta}=\frac{1}{L}\left(\int_{0}^{L}\left(\theta_{\left(x\right)}^{sim}-\theta_{\left(x\right)}^{exp}\right)^{2}dx\right)^{\frac{1}{2}}
\]
Error measures for each quantity are then defined 

\section{Parametrization of the Skin Friction Relation}

The skin friction closure relation is a function of the form:
\[
C_{f}=f_{\left(H,Re_{\theta}\right)}^{C_{f}^{0}}
\]
This function has three key features:
\begin{enumerate}
\item It tends to infinity as H approaches one:
\[
\lim_{H\rightarrow1}C_{f}=+\infty
\]
\item It is positive for share factors below separation $H<H_{sep}$ and
negative afterwards:
\[
\exists H_{sep}=f_{\left(Re_{\theta}\right)}\::\:\begin{cases}
C_{f}>0 & H<H_{sep}\\
C_{f}<0 & H>H_{sep}
\end{cases}
\]
\item And it tends to a nearly constant value in deep separation:
\[
\exists C_{f}^{dsep}=f_{\left(Re_{\theta}\right)}\::\:\lim_{H\rightarrow\infty}C_{f\left(H,Re_{\theta}\right)}=C_{f}^{dsep}
\]
\end{enumerate}
Values of the shape factor at which separation occurs $\left(H_{sep}\right)$
and the skin friction in deeply separated $C_{f}^{dsep}$ flows are
not precisely known. A good parametrization must let them be varied
by the minimization algorithm, and that is indirectly achieved by
the following  expression:
\[
C_{f\left(H,Re_{\theta},A_{i}\right)}^{mod}=S_{\left(H,Re_{\theta},A_{i}\right)}^{dim}\left(f_{\left(H,Re_{\theta}\right)}^{C_{f}^{0}}+\delta^{C_{f}}\right)-\delta^{C_{f}}\text{\qquad,\quad\ensuremath{\delta^{C_{f}}\in\mathbb{R}}}
\]
Where $\delta^{C_{f}}$ is a constant property of the parametrization
that enables variations in the shape factor at which the onset of
separation occurs, whereas the general shape of the closure relation
is taylored by the $S_{\left(\right)}^{dim}$. The dimensional shape
function maps the region for which the closure relation is being taylored
into a unit square, over which shape of the curve is taylored with
a classic Bernstein polynomial approach. 
\[
S_{\left(H,Re_{\theta},A_{i}\right)}^{dim}=B_{\left(\eta_{\left(H\right)}^{H},\eta_{\left(Re_{\theta}\right)}^{RT},A_{i}\right)}^{dxx}
\]
\[
\qquad with\quad\eta^{H}=\begin{cases}
0 & H<u_{ub}\\
\frac{H-H_{lb}}{H_{ub}-H_{lb}} & otherwise\\
1 & H>u_{ub}
\end{cases}\qquad and\quad\eta^{Re_{\theta}}=\begin{cases}
0 & Re<u_{ub}\\
\ln\left(\frac{Re}{Re_{min}}\right)/\ln\left(\frac{Re_{max}}{Re_{min}}\right) & otherwise\\
1 & H>u_{ub}
\end{cases}
\]

\section{Bernstein polynomials}

In the future, we will use Bernstein polynomials of arbitrary order.
However, for this early phase of the work we will restrict our study
to Bernstein polynomials of order 6 and degree 5. The basis of Bernstein
polynomials of order 6 and degree 5 is written as:
\[
\begin{array}{rl}
B_{\left(x\right)}^{d5r0} & =x^{5}\\
B_{\left(x\right)}^{d5r1} & =5x^{4}\left(1-x\right)\\
B_{\left(x\right)}^{d5r2} & =10x^{3}\left(1-x\right)^{2}\\
B_{\left(x\right)}^{d5r3} & =10x^{2}\left(1-x\right)^{3}\\
B_{\left(x\right)}^{d5r4} & =5x\left(1-x\right)^{4}\\
B_{\left(x\right)}^{d5r5} & =\left(1-x\right)^{5}
\end{array}\qquad\qquad\begin{array}{rl}
\frac{\partial}{\partial x}\left(B^{d5r0}\right) & =5x^{4}\\
\frac{\partial}{\partial x}\left(B^{d5r1}\right) & =5\left(4x^{3}\left(1-x\right)-x^{4}\right)\\
\frac{\partial}{\partial x}\left(B^{d5r2}\right) & =10\left(3x^{2}\left(1-x\right)^{2}-2x^{3}\left(1-x\right)\right)\\
\frac{\partial}{\partial x}\left(B^{d5r3}\right) & =10\left(2x\left(1-x\right)^{3}-3x^{2}\left(1-x\right)^{2}\right)\\
\frac{\partial}{\partial x}\left(B^{d5r4}\right) & =5\left(\left(1-x\right)^{4}-4x\left(1-x\right)^{3}\right)\\
\frac{\partial}{\partial x}\left(B^{d5r5}\right) & =-5\left(1-x\right)^{4}
\end{array}
\]
Let us now define the 6 bernstein polynomial (shape) coefficients:
\[
A_{1},A_{2},A_{3},A_{4},A_{5},A_{6}
\]
Used to construct arbitrary polynomials by linear combination of the
Bernstein basis:
\[
B_{\left(x,A_{i}\right)}^{d5}=A_{1}B_{\left(x\right)}^{d5r0}+A_{2}B_{\left(x\right)}^{d5r1}+A_{3}B_{\left(x\right)}^{d5r2}+A_{4}B_{\left(x\right)}^{d5r3}+A_{5}B_{\left(x\right)}^{d5r4}+A_{6}B_{\left(x\right)}^{d5r5}
\]
The derivative of $B^{d5}$ to the $x$-coordinate is written as:
\[
\frac{\partial}{\partial x}\left(B_{\left(x,A_{i}\right)}^{d5}\right)=A_{1}\frac{\partial}{\partial x}\left(B_{\left(x\right)}^{d5r0}\right)+A_{2}\frac{\partial}{\partial x}\left(B_{\left(x\right)}^{d5r1}\right)+A_{3}\frac{\partial}{\partial x}\left(B_{\left(x\right)}^{d5r2}\right)+A_{4}\frac{\partial}{\partial x}\left(B_{\left(x\right)}^{d5r3}\right)+A_{5}\frac{\partial}{\partial x}\left(B_{\left(x\right)}^{d5r4}\right)+A_{6}\frac{\partial}{\partial x}\left(B_{\left(x\right)}^{d5r5}\right)
\]

\section{Shape Function}

For now, we will only apply changes as a function of the shape factor
$H$. Let us then write:
\[
S_{\left(H,Re_{\theta},A_{i}\right)}^{dim}=B_{\left(\eta_{\left(H\right)}^{H},A_{i}\right)}^{d5}\qquad with\qquad\qquad with\quad\eta^{H}=\begin{cases}
0 & H<u_{ub}\\
\frac{H-H_{lb}}{H_{ub}-H_{lb}} & otherwise\\
1 & H>u_{ub}
\end{cases}
\]
So that we can write:
\[
\frac{\partial S}{\partial H}=\frac{\partial}{\partial H}\left(B_{\left(\eta_{\left(H\right)}^{H},A_{i}\right)}^{d5}\right)=\frac{\partial}{\partial\eta}\left(B_{\left(\eta_{\left(H\right)}^{H},A_{i}\right)}^{d5}\right)\frac{\partial\eta}{\partial H}\qquad with\qquad x=\eta^{H}
\]
So:
\[
\frac{\partial}{\partial\eta}\left(B_{\left(\eta_{\left(H\right)}^{H},A_{i}\right)}^{d5}\right)=\frac{\partial}{\partial x}\left(B_{\left(x,A_{i}\right)}^{d5}\right)
\]
And:
\[
\frac{\partial\eta}{\partial H}=\begin{cases}
\frac{\partial}{\partial H}\left(0\right) & H<u_{ub}\\
\frac{\partial}{\partial H}\left(\frac{H-H_{lb}}{H_{ub}-H_{lb}}\right) & otherwise\\
\frac{\partial}{\partial H}\left(1\right) & H>u_{ub}
\end{cases}=\begin{cases}
0 & H<u_{ub}\\
\frac{1}{H_{ub}-H_{lb}} & otherwise\\
0 & H>u_{ub}
\end{cases}
\]
Whereby:
\[
\frac{\partial S}{\partial H}=\frac{\partial}{\partial x}\left(B_{\left(x,A_{i}\right)}^{d5}\right)\frac{\partial\eta}{\partial H}
\]

\section{Cf Parametrization Summary}

The bernstein polynomials are given as:
\[
B_{\left(x,A_{i}\right)}^{d5}=A_{1}B_{\left(x\right)}^{d5r0}+A_{2}B_{\left(x\right)}^{d5r1}+A_{3}B_{\left(x\right)}^{d5r2}+A_{4}B_{\left(x\right)}^{d5r3}+A_{5}B_{\left(x\right)}^{d5r4}+A_{6}B_{\left(x\right)}^{d5r5}
\]
Their derivatives come as:
\[
\frac{\partial}{\partial x}\left(B_{\left(x,A_{i}\right)}^{d5}\right)=A_{1}\frac{\partial}{\partial x}\left(B_{\left(x\right)}^{d5r0}\right)+A_{2}\frac{\partial}{\partial x}\left(B_{\left(x\right)}^{d5r1}\right)+A_{3}\frac{\partial}{\partial x}\left(B_{\left(x\right)}^{d5r2}\right)+A_{4}\frac{\partial}{\partial x}\left(B_{\left(x\right)}^{d5r3}\right)+A_{5}\frac{\partial}{\partial x}\left(B_{\left(x\right)}^{d5r4}\right)+A_{6}\frac{\partial}{\partial x}\left(B_{\left(x\right)}^{d5r5}\right)
\]
The map from x to H is given as:
\[
x=\begin{cases}
0 & H<u_{ub}\\
\frac{H-H_{lb}}{H_{ub}-H_{lb}} & otherwise\\
1 & H>u_{ub}
\end{cases}
\]
The jacobian of the transformation is given as:
\[
\frac{\partial x}{\partial H}=\begin{cases}
0 & H<u_{ub}\\
\frac{1}{H_{ub}-H_{lb}} & otherwise\\
0 & H>u_{ub}
\end{cases}
\]
The dimensional shape function is given as:
\[
S_{\left(H,A_{i}\right)}^{d5}=B_{\left(\eta_{\left(H\right)}^{H},A_{i}\right)}^{d5}
\]
And it derivative is given as:
\[
\frac{\partial S^{d5}}{\partial H}=\frac{\partial}{\partial x}\left(B_{\left(x,A_{i}\right)}^{d5}\right)\frac{\partial x}{\partial H}
\]
The new skin friction correlation is given as:
\[
C_{f\left(H,Re_{\theta},A_{i}\right)}^{mod}=S_{\left(H,A_{i}\right)}^{d5}\left(f_{\left(H,Re_{\theta}\right)}^{C_{f}^{0}}+\delta^{C_{f}}\right)-\delta^{C_{f}}\text{\qquad,\quad\ensuremath{\delta^{C_{f}}\in\mathbb{R}}}
\]
Its derivative to H is given as:
\[
\frac{\partial}{\partial H}\left(C_{f\left(H,Re_{\theta},A_{i}\right)}^{mod}\right)=\frac{\partial S^{d5}}{\partial H}\left(f_{\left(H,Re_{\theta}\right)}^{C_{f}^{0}}+\delta^{C_{f}}\right)+S_{\left(H,A_{i}\right)}^{d5}\frac{\partial}{\partial H}\left(f_{\left(H,Re_{\theta}\right)}^{C_{f}^{0}}\right)
\]
Derivatives to other variables are simpler. Here, the example for
$Re_{\theta}$:
\[
\frac{\partial}{\partial Re_{\theta}}\left(C_{f\left(H,Re_{\theta},A_{i}\right)}^{mod}\right)=S_{\left(H,A_{i}\right)}^{d5}\frac{\partial}{\partial Re_{\theta}}\left(f_{\left(H,Re_{\theta}\right)}^{C_{f}^{0}}\right)
\]
Done!

\subsection{For paper}

The skin friction correlation is parametrized by combining the original
correlation $\left(C_{f\left(H,Re_{\theta}\right)}^{0}\right)$ with
a shape function $\left(S_{\left(H,A_{i}\right)}^{dM}\right)$ whose
shape depends on a set of coefficients $\left(A_{i}\right)$ that
are varied by the optimization algorithm:
\[
C_{f\left(H,Re_{\theta},A_{i}\right)}^{mod}=S_{\left(H,A_{i}\right)}^{dM}\left(f_{\left(H,Re_{\theta}\right)}^{C_{f}^{0}}+\delta^{C_{f}}\right)-\delta^{C_{f}}\text{\qquad,\quad\ensuremath{\delta^{C_{f}}\in\mathbb{R}}}
\]
The shape function consists of a linear combination of 5th degree
Bernstein polynomials and $\left(\delta^{C_{f}},\,H_{lb},\,H_{ub}\right)$
are constants that affect the parametrization scope:
\[
B_{\left(x,A_{i}\right)}^{dM}=\sum_{i=0}^{i=M}A_{i+1}B_{\left(x\right)}^{5i}\qquad with\qquad\left\{ \begin{array}{l}
B_{\left(x\right)}^{Mi}=\left(\begin{array}{c}
M\\
i
\end{array}\right)x^{i}\left(1-x\right)^{M-i}\\
x=\frac{H-H_{lb}}{H_{ub}-H_{lb}}
\end{array}\right.
\]
Setting all shape coefficients to unity yields the original closure
relation and the $H_{\left(H,Re_{\theta}\right)}^{*}$ correlation
was parametrized with a similar procedure.

\section{Hstar Parametrization}

The key feature here is to preserve the limit value for a collapsing
boundary layer:
\[
\lim_{H\rightarrow1}H^{*}=2
\]
This is easily by setting the lower limit of the $H$ intervention
region at $H_{min}=1$ (which we would do anyway), fixing the first
Bernstein coefficient of to $A_{1}^{H}=1$ (it is simply not provided
as a free parameter to the optimizer) and making a classic CST approach
(no offset needed here). So we write:
\[
H_{\left(H,Re_{\theta},A_{i}\right)}^{*mod}=S_{\left(H,A_{i}\right)}^{d5}\left(f_{\left(H,Re_{\theta}\right)}^{H^{*0}}\right)
\]
Whereby the derivative to $H$ comes from the chain rule:
\[
\frac{\partial}{\partial H}\left(H_{\left(H,Re_{\theta},A_{i}\right)}^{*mod}\right)=\frac{\partial}{\partial H}\left(S_{\left(H,A_{i}\right)}^{d5}\left(f_{\left(H,Re_{\theta}\right)}^{H^{*0}}\right)\right)
\]
\[
=\frac{\partial}{\partial H}\left(S_{\left(H,A_{i}\right)}^{d5}\right)\left(f_{\left(H,Re_{\theta}\right)}^{H^{*0}}\right)+S_{\left(H,A_{i}\right)}^{d5}\frac{\partial}{\partial H}\left(f_{\left(H,Re_{\theta}\right)}^{H^{*0}}\right)
\]
And the other derivatives come more simply, here the example for $Re_{\theta}$:
\[
\frac{\partial}{\partial Re_{\theta}}\left(H_{\left(H,Re_{\theta},A_{i}\right)}^{*mod}\right)=\frac{\partial}{\partial Re_{\theta}}\left(S_{\left(H,A_{i}\right)}^{d5}\left(f_{\left(H,Re_{\theta}\right)}^{H^{*0}}\right)\right)=S_{\left(H,A_{i}\right)}^{d5}\frac{\partial}{\partial Re_{\theta}}\left(f_{\left(H,Re_{\theta}\right)}^{H^{*0}}\right)
\]

\subsection{For paper}
\end{document}
